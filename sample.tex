% pLaTeX文書
\documentclass[a4paper,dvipdfmx]{jsarticle}% 要ドライバ指定
\usepackage{bxslashcell}
\usepackage{array}% あった方が綺麗
\begin{document}

\section{行の縦幅が通常通りの場合}

この場合、\verb|\slashcell| の縦幅(\verb|height|)を
指定する必要はない。
以下の表の左上のセルの内容は以下の通りである:
\begin{quote}\small\begin{verbatim}
\slashcell{5zw}
\end{verbatim}\end{quote}

\begin{center}
  \begin{tabular}{|c||c|r|}
    \hline
    \slashcell{5zw} & 値段 & カロリー \\
    \hline\hline
    牛丼並盛 & 500円 & 600 kcal \\
    牛丼大盛 & 1,000円 & 800 kcal \\
    牛丼特盛 & 1,500円 & 1,000 kcal \\
    \hline
    牛皿並盛 & 300円 & 250 kcal \\
    牛皿大盛 & 700円 & 300 kcal \\
    牛皿特盛 & 1,000円 & 350 kcal \\
    \hline
  \end{tabular}
\end{center}

\section{当該行に複数行テキストのセルがある場合}

\verb|\slashcell| のあるセルと同じ行に \verb|p| 指定のセルが
存在する場合は、そのセルがもつテキストの(組版結果の)行数
(\verb|p| のセルが複数ある場合は最大の行数)を \verb|height|
パラメタに指定する。
以下の表の左上のセルの内容は以下の通りである:
\begin{quote}\small\begin{verbatim}
\slashcell[cross,height=2]{6zw}
\end{verbatim}\end{quote}

\begin{center}
  \renewcommand{\arraystretch}{1.2}
  \newcommand*{\xHC}[2]{%
    \multicolumn{1}{|p{4zw}|}{%
      \centering #1\\\mbox{\<(#2)\<}}}
  \begin{tabular}{|c||r|r|}
    \hline
    \slashcell[cross,height=2]{6zw} & \xHC{値段}{円}
      & \xHC{カロリー}{kcal} \tabularnewline
    \hline\hline
    牛丼並盛 & 500 & 600  \\
    牛丼大盛 & 1,000 & 800  \\
    牛丼特盛 & 1,500 & 1,000  \\
    \hline
    牛皿並盛 & 300 & 250  \\
    牛皿大盛 & 700 & 300  \\
    牛皿特盛 & 1,000 & 350  \\
    \hline
  \end{tabular}
\end{center}
\end{document}
